\documentclass[]{deedy-resume-openfont}

\begin{document}

% last updated date
\lastupdated


%title name
\namesection{}{John P. Ortiz}{ \urlstyle{same}\url{http://johnportiz.weebly.com} \\
\href{mailto:john.p.ortiz.14@gmail.com}{john.p.ortiz.14@gmail.com} | 541.207.5846}

%%%%%%%%%%%%%%%%%%%%%%%%%%%%%%%%%%%%%%%%%%%%%%%%%%%%%%
%
%	COLUMN ONE
%
%%%%%%%%%%%%%%%%%%%%%%%%%%%%%%%%%%%%%%%%%%%%%%%%%%%%%%

\begin{minipage}[t]{0.33\textwidth}

%%%%%%%%%%%%%%%%%%%%%%%%%%%%%%%%%%%%%%%%%%%%%%%%%%%%%%
%	EDUCATION
%%%%%%%%%%%%%%%%%%%%%%%%%%%%%%%%%%%%%%%%%%%%%%%%%%%%%%

\section{Education}

\subsection{New Mexico Tech}
\descript{M.Sc. Hydrology}
\location{May 2017 | Socorro, NM | 3.89 GPA}
Thesis: The role of fault-zone architectural elements and basal altered zones on downward pore pressure propagation and induced seismicity in the crystalline basement
\sectionsep

\subsection{Dartmouth College}
\descript{B.A. Earth Sciences (Honors)}
\location{June 2010 | Hanover, NH}
Honor's Thesis: Quantifying regional sediment flux from observations of nearshore morphology in the Columbia River Littoral Cell
\sectionsep


%%%%%%%%%%%%%%%%%%%%%%%%%%%%%%%%%%%%%%%%%%%%%%%%%%%%%%
%	COURSEWORK
%%%%%%%%%%%%%%%%%%%%%%%%%%%%%%%%%%%%%%%%%%%%%%%%%%%%%%

\section{Coursework}

Advanced Hydrological Modeling \\
Quantitative Hydrologic Methods \\
Contaminant Hydrogeology \\
Statistical Modeling \& Machine Learning \\
Continuum Fluid Dynamics \\
Petroleum Reservoir Engineering \\
Depositional Systems \& Basin Analysis \\
Subsurface and Petroleum Geology \\
Hydrogeochemistry \\
\sectionsep

\subsection{Short Courses}
MATLAB Reservoir Simulation Toolbox \\
PFLOTRAN Short Course \\
Reservoir Geomechanics \\
Pioneer Natural Resources Field Course

%%%%%%%%%%%%%%%%%%%%%%%%%%%%%%%%%%%%%%%%%%%%%%%%%%%%%%
%	SKILLS
%%%%%%%%%%%%%%%%%%%%%%%%%%%%%%%%%%%%%%%%%%%%%%%%%%%%%%

\section{Skills}
\subsection{Programming}
Python 
\textbullet{} MATLAB 
\textbullet{} R 
\textbullet{} Bash/shell
\textbullet{} \LaTeX

\subsection{Software}
FEHM 
\textbullet{} PFLOTRAN 
\textbullet{} MODFLOW
\textbullet{} Git 
\\ Petromod
\textbullet{} ArcGIS 
\textbullet{} Adobe Illustrator
\\ Unix \textbullet{} Linux \textbullet{} MacOS X \textbullet{} Windows
\\ COMSOL Multiphysics

%%%%%%%%%%%%%%%%%%%%%%%%%%%%%%%%%%%%%%%%%%%%%%%%%%%%%%
%	SOCIETIES
%%%%%%%%%%%%%%%%%%%%%%%%%%%%%%%%%%%%%%%%%%%%%%%%%%%%%%

\section{Societies}
AGU\\
GSA\\
AAPG\\
NGWA\\
InterPore
\sectionsep

%%%%%%%%%%%%%%%%%%%%%%%%%%%%%%%%%%%%%%%%%%%%%%%%%%%%%%
%
%	COLUMN TWO
%
%%%%%%%%%%%%%%%%%%%%%%%%%%%%%%%%%%%%%%%%%%%%%%%%%%%%%%
\end{minipage}
\hfill
\begin{minipage}[t]{0.60\textwidth}

%%%%%%%%%%%%%%%%%%%%%%%%%%%%%%%%%%%%%%%%%%%%%%%%%%%%%%
%	EXPERIENCE
%%%%%%%%%%%%%%%%%%%%%%%%%%%%%%%%%%%%%%%%%%%%%%%%%%%%%%

\section{Experience}

\runsubsection{Los Alamos National Lab}
\descript{| Post-Master's Researcher }
\location{June 2017 - present | Los Alamos, NM}
\vspace{\topsep} % hacky fix for awkward extra vertical space
\begin{tightemize}
	\item Simulating radionuclide gas transport in fractured geologic media using multiple finite-element method (FEM) and control volume finite-element (CVFEM) numerical models.
	\item Determining field-scale transport properties of rocks using models and tracer experiments.
	\item Developing the Amanzi high performance computing (HPC) flow \& transport simulator to meet the Nuclear Quality Assurance-1 (NQA-1) regulatory standard by improving code verification and benchmark tests.
	\item Developing a reduced-order model (ROM) for rapid prediction of gas seepage times.
\end{tightemize}
\sectionsep

\runsubsection{New Mexico Tech}
\descript{| Graduate RA/TA }
\location{August 2015 - May 2017 | Socorro, NM}
%\vspace{\topsep} % hacky fix for awkward extra vertical space
\begin{tightemize}
	\item Created transient 3D finite-difference (FDM) models in MODFLOW to analyze fluid-fault interactions as pertaining to a suite of basal reservoir injection scenarios. 
	\item I also developed transient 2D cross-sectional FDM models in MATLAB to test fluid-fault interactions for crystalline basement fault zones exhibiting local, dynamically enhanced permeability caused by excess fluid pressures.
	\item Deployed subsurface field survey equipment (transverse electromagnetics [TEM], magnetotellurics [MT]) to interpret deep saline geothermal flow regimes in order to evaluate potential hydrothermal systems in southern New Mexico.
\end{tightemize}
\sectionsep

\runsubsection{Oregon State University}
\descript{| Research Intern }
\location{June 2013 - September 2013 | Corvallis, OR}
%\vspace{\topsep} % hacky fix for awkward extra vertical space
\begin{tightemize}
	\item Collected nearshore topographic and bathymetric data.
	\item Interpolated, and visualized nearshore bathymetry in order to extract key spatial and temporal metrics using MATLAB.
\end{tightemize}
\sectionsep

\runsubsection{US Army Corps of Engineers}
\descript{| Research Intern }
\location{January 2013 - March 2013 | Duck, NC}
%\vspace{\topsep} % hacky fix for awkward extra vertical space
\begin{tightemize}
	\item Collected and processed LiDAR observations during and after storm surges to monitor coastal evolution and erosion.
	\item Created a paleo-hurricane record of St. Croix using grain-size analysis and on sediment cores combined with charcoal carbon dating.
\end{tightemize}
\sectionsep


%%%%%%%%%%%%%%%%%%%%%%%%%%%%%%%%%%%%%%%%%%%%%%%%%%%%%%
%	OTHER
%%%%%%%%%%%%%%%%%%%%%%%%%%%%%%%%%%%%%%%%%%%%%%%%%%%%%%

\section{Awards}
LANL "Spot" Performance Award

	
\end{minipage}
\end{document} \documentclass[]{article}
	





